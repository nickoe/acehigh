%
% $Id$
%

\documentclass[11pt,a4paper,oneside]{memoir}

\usepackage[utf8]{inputenc}
\usepackage[T1]{fontenc}
\usepackage[adobe-utopia]{mathdesign}
\usepackage[scaled]{luximono}

\usepackage[nofancy]{svninfo}
%\usepackage{fancyhdr}
\svnInfo $Id$

\usepackage[draft]{fixme}


% memoir kan ikke selv håndtere dette (måske med captionof)
%\RequirePackage{caption}
%\captionsetup{font=small,labelfont={bf,sf}}

\captionnamefont{\sffamily \bfseries}
\captiontitlefont{}


\usepackage[colorlinks]{hyperref}

\chapterstyle{section}
\setsecnumdepth{section}
% TODO: Set up the ToC to list subsections a bit more elegant

\title{AceHigh Control Protocol}
\date{Revision \svnInfoRevision}

\begin{document}

\maketitle

\tableofcontents*

\chapter{Introduction}

Why this paper? Why is it necessary? How should it be read?

This paper describes the AceHigh Control Protocol used for controlling
an AceHigh plotter.

This protocol is influenced by the SD protocol and the RepRap
Generation 3 Protol (see
\url{http://docs.google.com/Doc?id=dd5prwmp_14ggw37mfp&hl=en}).

This is a draft and everything in this paper is subject to change.


\section{Terminology}

The host referes to the PC running the host software, whereas the
device referes to the microcontroller.


\chapter{Protocol description}


\section{Communication model}

The communication between the host and the device is a bit SPI
like. The host sends a request, and the device responds to this
request.

The host is not allowed to send a request before the response from the
previous request has been recieved.


\section{Packet structure}

As shown in figure~\ref{fig:request-packet}, a request packet consists
of a synchronization byte, a command byte, parameter bytes, and a CRC
byte.

\begin{figure}[htbp]
  \centering
  % Synchonization byte, command byte, parameters (depends on
  % command), CRC-8-Dallas/Maxim entire packet excluding CRC byte
  % itself
  \vspace{2cm}
  \caption{Structure of a request packet}
  \label{fig:request-packet}
\end{figure}\fixme{make drawing}

The synchronization byte has the value of $\textup{D}5_{16}$ (or
$213_{10}$ og $11010101_2$). A list of available command bytes can be
found in section~\ref{sc:commands}. The number of parameter bytes
depends on the command; some commands has no parameter bytes and the
byte just after the command byte will be the CRC byte. The CRC byte is
the CRC-8-Dallas/Maxim of the entire packet (including synchronization
byte, excluding the CRC byte itself).


\begin{figure}[htbp]
  \centering
  % Response header, parameters (depends on command),
  % CRC-8-Dallas/Maxim of header and parameters IF parameters exist
  \vspace{2cm}
  \caption{Structure of responce packet}
  \label{fig:response-packet}
\end{figure}\fixme{make drawing}


\section{Commands}
\label{sc:commands}
% here: a list of available commands and their parameters and expected
% responses

\subsection{CMD0 - Exchange version information}

This commands sends the version information of the host to the device,
and the revice respondes with it's own version information.

\begin{table}[htbp]
  \centering
  \caption{Parameters and response for CMD0}
  \label{tab:cmd0}
  \begin{tabular}{lcc}
    \toprule
    & \textbf{Number of bytes} & \textbf{Interpretation} \\
    \midrule
    Parameters & 2 & TBD \\
    Response & 2 & TBD \\
    \bottomrule
  \end{tabular}
\end{table}


\section{Error detection and correction}

This protocol relies on CRC matches and time-outs to ensure, that the
data is recieved unaltered.


\subsection{CRC mismatch}

Each request packet and some response packets contains a CRC byte. If
the command and parameter bytes does not match this CRC, the packet is
considered corrupted.

Next action depends on which node detects the error:
\begin{itemize} \firmlist
\item If \textit{the device detects} an error, it will set the error
  bit in the response byte.
\item if \textit{the host detects} an error, it should repeat the
  request or cause a time-out to reset the connection.
\end{itemize}

\subsection{Time-out}

If a node does not recieve an expected byte within the allowed time,
the connection is considered lost and will be reset, and the node
should clean the packet buffer. The device should still be configured
for the current plotting session, but the connection is considered
reset.

\fixme{allowed time to be decided}


\chapter{Revision history}

This section will be written when the first stable version of this
protocol is designed.


\end{document}

