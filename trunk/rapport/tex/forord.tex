\chapter{Forord}

% bl.a. tak til David Pedersen (Kjærgaards fætter) for vejledning om
% hvad? RepRap-fællesskabet (specielt fenn)

% Terminologi, om kodeeksempler, rapportopbygning, kildehenvisning,
% referencer

Tak til miljøet omkring den frie software. Fri tale, gratis øl og alt
det der\dots \fixme{Skriv kort opbygningen af rapporten - Det at vi
  har en design og implementeringsdel}

\section{Terminologi}

% Her skriver vi hvilken betydning forskellige ord har i
% rapporten. Hvad betyder fx software, MCU, plotter etc.

Med software menes den ekserkvering, der sker på mikroprocessoren i
plotteren. Når forkortelsen \enquote{MCU} nævnes, menes der selve
mikroprocessoren. MCU er en forkortelse for \textit{MicroController
  Unit}. Hardwarediagram er selve diagrmmet over de elektroniske
kredsløb, mens printudlægget er selve det færdige kredsløb på
printkortet.
\fixme{Terminologi sektion renskrives!}

\section{Om kodeeksempler i rapporten}

Alt software i vores projekt er udviklet med den frie GNU AVR C
Compiler\fixme{korrektur}, som virker på både Windows og Linux.

%%% Local Variables: 
%%% mode: latex
%%% TeX-master: "../master"
%%% End: