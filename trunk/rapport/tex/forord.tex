\chapter{Forord}
\mnote{Fælles}
% Terminologi, om kodeeksempler, rapportopbygning, kildehenvisning,
% referencer

\section{Om rapportens opbygning}

Rapporten er delt ind i fire hovedafsnit.

Afsnittet~\titleref{ch:indledning} på side~\pageref{ch:indledning}
beskriver de krav, plotteren skal overholde. Afsnittet giver et
udgangspunkt for at lave en løsning.

Delen~\titleref{prt:design} på side~\pageref{prt:design} diskuterer
forskellige løsninger til de krav, der er opstillet i
indledningen. Delen diskuterer mulige mekaniske, hardwaremæssige og
softwaremæssige løsninger og begrunder valget af den løsning, der
implementeres.

Delen~\titleref{prt:implementering} på
side~\pageref{prt:implementering} beskriver, hvordan den valgte
løsning realiseres og gennemgår i detaljer, hvordan løsningen
fungerer.

Afsnittet~\titleref{ch:afslutning} på side~\pageref{ch:afslutning}
samler op på vores projekt ved vurdering af vores færdige produkt og
perspektivering.

Bilaget~\titleref{ch:bilag-rettelser-software} på
side~\pageref{ch:bilag-rettelser-software} indeholder rettelser til
dokumentationen af softwaredesignet og -implementeringen.


\section{Terminologi}

% Her skriver vi hvilken betydning forskellige ord har i
% rapporten. Hvad betyder fx software, MCU, plotter etc.

Med software menes den ekserkvering, der sker på mikroprocessoren i
plotteren. Når forkortelsen \enquote{MCU} nævnes, menes der selve
mikroprocessoren. MCU er en forkortelse for \textit{Micro Controller
  Unit}. Hardwarediagram er selve diagrammet over de elektroniske
kredsløb, mens printudlægget er selve det færdige kredsløb på
printkortet. Step og skridt bruges i flæng, når vi omtaler at en
stepmotor skifter position.

Vi bruger forkortelsen gu for graphics units, som er $\tfrac 1{40}$
mm. 1 gu = 0,025 mm.


\section{Angivelse af forfatter}

For at efterkomme et formelt krav er navnet på forfatteren til det
enkelte afsnit angivet i margin ved afsnittets begyndelse.


\section{Kilder}
Til skabelsen af plotteren har vi selvfølgelig fået inspiration og
information fra andre kilder end os selv. Vi har derfor opstillet en
litteraturliste, som det ses på side~\pageref{ch:litteratur}, med den
litteratur og de kilder, vi har brugt.

Hver litteraturelement har et nummer, som vi referer til i rapporten
ved at skrive [nummer]. Fx vil [1] refere til det første element i
litteraturlisten.


\section{CD til rapporten}

Med rapporten følger en cd med rapporten i PDF-format, en video af et
plot og kildekoderne til softwaren på plotteren.

Kildekoderne kan kompileres med GNU AVG-GCC (el. WinAVR) og GNU Make.


\section{Takkeskrivelse}

% bl.a. tak til David Pedersen (Kjærgaards fætter) for vejledning om
% hvad? RepRap-fællesskabet (specielt fenn)
% Måske takke nogle af de kilder der er nævnt i litteraturlisten

Vi vil gerne rette en tak til alle, der på den ene eller anden måde
har bidraget til projektets tilbliven. Her særligt tak til
RepRap-fællesskabet\footnote{Se \url{http://www.reprap.org}.} og
\texttt{fenn}\footnote{Ukendt RepRap-fan mødt på IRC.} for
inspiration. Tak til David S. Pedersen for vejledning i forbindelse
med valg af dataformat. Tak til Roland Riegel\footnote{Se
  \url{http://www.roland-riegel.de}.} for GPL-software til
SD-kort. Tak til Bent Erik Mäkinen Thomsen for hjælp og vejledning
gennem projektforløbet.



{ \centering

  \vspace{3cm}

  \begin{minipage}{0.7\textwidth}
    \hrule 
    \vspace{1mm}
    Christian Klim Hansen
    \vspace{1.5cm}
    \hrule
    \vspace{1mm}
    Kristian Kjærgaard
    \vspace{1.5cm}
    \hrule
    \vspace{1mm}
    Nick Østergaard
  \end{minipage}

}


%%% Local Variables: 
%%% mode: latex
%%% TeX-master: "../master"
%%% End: