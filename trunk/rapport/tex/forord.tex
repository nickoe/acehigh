\chapter{Forord}

% Terminologi, om kodeeksempler, rapportopbygning, kildehenvisning,
% referencer

\section{Om rapportens opbygning}

Rapporten er delt ind i fire hovedafsnit.

Afsnittet~\titleref{ch:indledning} på side~\pageref{ch:indledning}
beskriver de krav, plotteren skal overholde. Afsnittet giver et
udgangspunkt for at lave en løsning.

Delen~\titleref{prt:design} på side~\pageref{prt:design} diskuterer
forskellige løsninger til de krav, der er opstillet i
indledningen. Delen diskuterer mulige mekaniske, hardwaremæssige og
softwaremæssige løsninger og begrunder valget af den løsning, der
implementeres.

Delen~\titleref{prt:implementering} på
side~\pageref{prt:implementering} beskriver, hvordan den valgte
løsning realiseres og gennemgår i detaljer, hvordan løsningen
fungerer.

Afsnittet~\titleref{ch:afslutning} på side~\pageref{ch:afslutning} samler op på vores projekt.
Afsnittet hænger sammen med kravspecifikationen. Har vi opfyldt kravene? Hvad har vi lært?
Hvad kunne vi gøre bedre?
\fixme{Anden formulering?}

\section{Terminologi}

% Her skriver vi hvilken betydning forskellige ord har i
% rapporten. Hvad betyder fx software, MCU, plotter etc.

Med software menes den ekserkvering, der sker på mikroprocessoren i
plotteren. Når forkortelsen \enquote{MCU} nævnes, menes der selve
mikroprocessoren. MCU er en forkortelse for \textit{MicroController
  Unit}. Hardwarediagram er selve diagrammet over de elektroniske
kredsløb, mens printudlægget er selve det færdige kredsløb på
printkortet.
\fixme{Terminologi sektion renskrives!}

\fixme{step og skridt bruges i flæng om når en stepmotor skifter
  position}

\section{Kilder}
Til skabelsen af plotteren, har vi selvfølgelig fået inspiration og
information fra andre kilder end os selv. Vi har derfor opstillet en
litteraturliste, som det ses på side ~\pageref{ch:litteratur}, med den
litteratur og de kilder vi har brugt.

\section{Takkeskrivelse}

% bl.a. tak til David Pedersen (Kjærgaards fætter) for vejledning om
% hvad? RepRap-fællesskabet (specielt fenn)
% Måske takke nogle af de kilder der er nævnt i litteraturlisten

Tak til miljøet omkring den frie software. Fri tale, gratis øl og alt
det der\dots \fixme{lav takkeskrivelse}


%%% Local Variables: 
%%% mode: latex
%%% TeX-master: "../master"
%%% End: