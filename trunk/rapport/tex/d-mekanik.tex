\chapter{Mekanik}

% Her beskrives de overvejelser vi har haft omkring udformningen af
% vores mekanik.

I vores overvejeler omkring udformningen af vores produkt, så vi på
selve funktionen af produktet.  Da vores produktet har til formålet at
plotte en figur på et stykke A4-papir, så var det vigtigt, at selve
plotningen foregik uden slør. For at undgå dette så vi på placeringen
af pennen. Vi havde to forskellige forslag til udformningen
glideren. De to stænger, som penne kører på, kunne enten være placeret
vandret, som ses på figur\fixme{indsæt figur af glider}, eller
lodret. Vi kom frem til, at det, som ville resultere i mindst slør,
var, hvis gliderne var placeret vandret. Pennen styres af en
stepmotor, som er placeret ude i en af siderne.


Et andet problem omkring slør var placeringen af stepmotoren, som skal
styre gliderne. Her kunne vi placere stepmotoren i midten af selve
konstruktionen eller ude i en af siderne \fixme{indsæt evt.  en
  håndlavet tegning af stepmotor, som styrer glideren}. Hvis
stepmotoren blev placeret i midten af plotteren, så ville det først og
fremmest resultere i en større konstruktion, da stepmotoren skulle
placeres over glideren, men ligeledes ikke gøre meget for at hjælpe på
sløret. Hvis vi istedet placere stepmotoren i siden, så vil den være i
samme plan som glideren, hvilket ville gøre konstruktionen en del
mindre, men så hjælpe på sløret. Glideren ville da først bevæge sig,
når sløret er nogenlunde væk, hvilket giver anledning til en mere
præcis plotning.

Enkelheden af konstruktionen var ligeledes en essentiel del af vores
projekt, da vi har en begrænset periode, hvori vi kan udforme vores
produkt. Produktets størrelser var dog ikke under magen diskution, da
det simplistiske skulle være en gennemtrængende del uden at have en
indvirkning på produktets funktionalitet. Det skal bemærkes, at
konstruktionen næsten udelukkende er lavet af PVC, da det er billigt
og let at arbejde med til forskel for en konstruktion lavet af
stål. Gliderne samt glidelejrene er lavet i stål for reduceret
gnidningsmodstand. Vi bruger også olie til yderligere at gøre
modstanden, og derved sløret, mindre.


%%% Local Variables: 
%%% mode: latex
%%% TeX-master: "../master"
%%% End: 