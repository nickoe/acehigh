\chapter{Rettelser til softwaredesign og -implementering}
\label{ch:bilag-rettelser-software}

% Vi havde problemer med vores software - her skriver vi hvordan vi
% ende med at designe og implementere vores software

Grundet tekniske problemer er softwaren designet og implementeret
anderledes end tidligere beskrevet. Dette bilag indeholder rettelser
til dokumentationen af softwaredesignet og -implementeringen.

\section{Rettelser til motorkontrol}

\subsection{Design}
\label{ssc:rettelse-design}

Før interesserede vi os for det næste tidspunkt, motoren skal flyttes
på. Vi itererede $t(n_x)$ på $n_x$.

Nu interesserer vi os for hvilket skridt, motoren skal stå på på
nuværende tidspukt. Vi itererer $n_x(t)$ på $t$. Hver iteration
foregår med samme frekvens som før, så vi kan højst flytte motoren ét
step pr. iteration. Det betyder at sammenhængen
\begin{align}
  n_x(t+1) - n_x(t) \leq 1
\end{align}
gælder og skal overholdes ved ikke at køre for hurtigt.

I stedet for som i sætning\vref{eq:t-af-nxdnxlv} at være interesseret i et
tidspunkt $t$, er vi interesseret i et step $n_x$:
\begin{align}
  n_x = t \times \frac{\Delta n \times L}v
\end{align}

Det ses, at størrelsen $\frac{\Delta n \times L}v$ ikke varierer for
en linie. Derfor indføres størrelsen $k_x$ som
\begin{align}
  k_x &= \frac{\Delta n \times L}v \Rightarrow \\
  n &= t \times k_x
\end{align}
og beregnes inden linien tegnes.


\subsection{Implementering}
\label{ssc:rettelse-implementering}

Symbolerne $k_x$ og $k_y$ beregnes og iterationsprocessen
startes. Mens iterationsprocessen kører, beregnes $k_x$ og $k_y$ for
næste linie. Så snart iterationsprocessen er færdig, startes den igen
med $k_x$ og $k_y$ for næste linie. Se afviklingsdiagram i
figur\vref{fig:rettelser-afvikling}.

\begin{figure}[htbp]
  \centering
  \subfloat[Afviklingsdiagram over den del af motorkontrollen, der
  afvikles efter behov]{
    \includegraphics[width=0.40\textwidth]{./img/egentlig-softwimpl-a}
  }
  \qquad
  \subfloat[Afviklingsdiagram over realtidsdelen af
  motorkontrollen. Timeren starter afviklingen periodisk.]{
    \includegraphics[width=0.40\textwidth]{./img/egentlig-softwimpl-b}
  }
  \caption{Egentligt afviklingsdiagram for implementering af
    motorkontrol}
  \label{fig:rettelser-afvikling}
\end{figure}

Iterationsprocessen styres af MCU'ens timer. Når en iteration
foretages, afbrydes den almindelige afvikling så længe iterationen
foregår. Når iterationen er færdig, genoptages den almindelige
afvikling, indtil timeren løber ud igen og det igen er tid til at
foretage en iteration.


\subsection{Konsekvens af rettelser}

Hvis der tegnes mange små linier, kan MCU'en have problemer med at
udføre de nødvendige behandlinger, før tegnehovedet er ankommet til
liniens endepunkt og går igang med at tegne den nye linie. Det
betyder, at tegnehovedet står stille på liniens endepunkt mens den
venter på instruktioner til næste linie.

Det ønskede design kan let kompencere for dette. Køen fyldes med
instruktioner og tegnehovedet begynder at tegne. Når køen er ved at
være tømt, løftes tegnehovedet og venter til køen igen er fyldt. Det
aktuelle design kan ikke kompencere for dette, da designet ikke
indeholder en kø.

Den tilsyneladende eneste praktiske konsekvens ved ændret software er,
at den virker.


\section{Valg af programmeringssprog}
\label{sec:bilag-programmeringssprog}

Som led i fejlsøgning blev softwaren portet fra \Cpp\ til C som det
eneste sprog, softwaren er skrevet i. Desuden unødvendiggjorde det
egentlige design nødvendigheden af særlige egenskaber ved \Cpp .


%%% Local Variables: 
%%% mode: latex
%%% TeX-master: "../master"
%%% End: 
