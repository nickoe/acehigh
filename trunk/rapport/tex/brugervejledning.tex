\chapter{Brugervejledning}
\label{ch:brugervejledning}
\mnote{Nick Østergaard}

% Interface (kanp-oversigt)
% Overførsel af data, eventuelt generering af data
% Placering af papir
% Specifikationer - tegneområde, placering af et ark papir

For at tegne grafik i HPGL-formatet, gør følgende:

\begin{enumerate}[\quad \bfseries 1.] \firmlist
\item Konverter tegningen til HPGL (hvis programmet ikke kan
  konvertere til HPGL men til PostScript, kan
  \texttt{pstoedit}\footnote{Se \url{http://www.pstoedit.net}.})
\item Brug \texttt{HPGL Distiller}\footnote{Se
    \url{http://pldaniels.com/hpgl-distiller}.} på HPGL-tegniningen.
\item Overfør tegningen til SD-kortet med
  f.eks. \texttt{dd}\footnote{Se
    \url{http://en.wikipedia.org/wiki/Dd_(Unix)}.} eller
  \texttt{HxD}\footnote{Se
    \url{http://mh-nexus.de/en/hxd}.}. SD-kortet bruger \textit{ikke}
  et filsystem, så første byte i tegningen er første byte på
  hukokmmelsesblokken.
\item Sæt SD-kortet i plotteren og flyt langsomt tegnehovedet ned i
  hjørnet.
\item Tænd for strømmen til stepmotorerne, tegnehovedet og sidst
  MCU'en. Maskinen begynder at tegne.
\end{enumerate}

\fixme{Sammen: Hvad med at angive en kort beskrivelse af dd og hxd i
  stedet for wiki link?}

%%% Local Variables: 
%%% mode: latex
%%% TeX-master: "../master"
%%% End: 