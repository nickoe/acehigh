\chapter{Brugervejledning}
\label{ch:brugervejledning}
\mnote{Kristian Kjærgaard og Nick Østergaard}

% Interface (kanp-oversigt)
% Overførsel af data, eventuelt generering af data
% Placering af papir
% Specifikationer - tegneområde, placering af et ark papir


\section{Inkscape til HPGL}

\begin{enumerate}[\quad \bfseries 1.] \firmlist
\item Konverter din Inkscape tegning til \texttt{eps}.
\item Konverter tegningen til HPGL via \texttt{pstoedit}\footnote{Se \url{http://www.pstoedit.net}.} 
  \begin{enumerate}[-]
  \item \texttt{pstoedit -f plot-hpgl billedfil.eps billedfil.hpgl}
  \end{enumerate}
\item Brug \texttt{HPGL Distiller}\footnote{Se
    \url{http://pldaniels.com/hpgl-distiller}.} på HPGL-tegniningen
  som \texttt{pstoedit} generede.
  \begin{enumerate}[-]
  \item \texttt{hpgl-distiller -i billedfil.hpgl -o distilled.hpgl}
  \end{enumerate}
\end{enumerate}


\section{Plot HPGL-fil}

\begin{enumerate}[\quad \bfseries 1.] \firmlist

\item Overfør tegningen til SD-kortet med
  f.eks. \texttt{dd}\footnote{Se
    \url{http://en.wikipedia.org/wiki/Dd_(Unix)}.} eller
  \texttt{HxD}\footnote{Se
    \url{http://mh-nexus.de/en/hxd}.}. SD-kortet bruger \textit{ikke}
  et filsystem, så første byte i tegningen er første byte på
  hukokmmelsesblokken. Eksempel med \texttt{dd}:
  \begin{enumerate}[-]
  \item \texttt{dd if=sti/til/distilled.hpgl of=sti/til/sdkort}
  \end{enumerate}
\item Sæt SD-kortet i plotteren og flyt langsomt tegnehovedet ned i
  hjørnet.
\item Tænd for strømmen til stepmotorerne, tegnehovedet og sidst
  MCU'en. Maskinen begynder at tegne.
\end{enumerate}


%%% Local Variables: 
%%% mode: latex
%%% TeX-master: "../master"
%%% End: 