\chapter{Brugervejledning}
\label{ch:brugervejledning}

% Interface (kanp-oversigt)
% Overførsel af data, eventuelt generering af data
% Placering af papir
% Specifikationer - tegneområde, placering af et ark papir

For at tegne grafik i HPGL-formatet, gør følgende:

\begin{enumerate}[\quad \bfseries 1.]
\item Konverter tegningen til HPGL (hvis tegningen ikke er tegnet i et
  program, der kan eksportere til HPGL, kan \texttt{HPGL
    Distiller}\footnote{Se \url{http://pldaniels.com/hpgl-distiller}.}
  muligvis konvertere tegningen til HPGL).\fixme{Så vidt Nick læser
    linket, er det pstoedit du skal bruge for at lave HPGL filen, hvor
    HPGL-distillern kun gør den simplere}
\item Sørg for, at formatet overholder den implementering af formatet,
  projektet har brugt. Se afsnit hvad?\fixme{henvisning}
\item Overfør tegningen til SD-kortet med
  f.eks. \texttt{dd}\footnote{Se
    \url{http://en.wikipedia.org/wiki/Dd_(Unix)}.} eller
  \texttt{HxD}\footnote{Se
    \url{http://mh-nexus.de/en/hxd}.}. SD-kortet bruger ikke et
  filsystem, så tegningen skal ligge på \textit{blokken}.
\item Sæt SD-kortet i plotteren og tænd plotteren.
\item Klik på \enquote{Tegn}. Hvis der ikke er opstået problemer,
  begynder maskinen at tegne.
\end{enumerate}

\fixme{Sammen: Hvad med at angive en kort beskrivelse af dd og hxd i
  stedet for wiki link?}

%%% Local Variables: 
%%% mode: latex
%%% TeX-master: "../master"
%%% End: 