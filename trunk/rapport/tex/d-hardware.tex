\chapter[Hardwaredesign]{Hardware}

% Afsnittet skal indeholde information om vores
% hardware. Strømforsyning, motor drivere, MCU print, transducere
% m.m. Men kun hvad vi kan gøre, for at opfylde vores krav, ikke
% hvordan vi faktisk konstruerer det.

\section{Stepmotor drivere}
I dette projekt har vi to forskellige stepmotorer, som vi har fundet
fra et par bordscannere. Motorene er selvfølgelig forskellige fra
hinanden. Den ene motor er en bipolar motor mens den anden er en
variabel reluktans motor. Vi har i det følgende valgt, at lave hvert
driver print på hvert sit printkort, da vi opnår en vis fleksibilitet,
da vi på den måde lettere kan lave den ene om, i tilfælde af den anden
brænder af eller der er fejl på. \fixme{todo, Omformuler dårlige
  formuleringer}

\subsection{Bipolar motor}
\fixme{Nærbillede af lille motor i marginen, gerne monteret på plotter}

\subsection{Unipolar motor}
\fixme{Nærbillede af stor motor i marginen, gerne monteret på plotter}

\section{Sensorer}
Vi skal være i stand til at nulstille maskinens position i forhold til
vores koordinatsystem, fordi vi skal vide hvor vi har vores
udgangspunkt.

For at opnå dette kan man gøre det, at man kører motorerne i retning
mod et nulpunkt, men kører længere end det faktisk er muligt at
køre. Det skal have til formål, at hovedet bliver trukket til
origo. Men denne metode er ikke anvendelig, da man vil belaste
mekanikken og motorerne for meget.

Vi vil derfor anvende nogle sensorer som følrere i en
nulstillingsposition. Vi kan anvende et par fotogafler, hvor vi fører
en mekanisk fane ind i. Maskinen kan så dedektere, hvornår der kommer
noget ind for sensoren, hvilket betyder, at vi ved hvor vi er. Ud fra
hvilke signaler vi får fra de forskellige sensorer, så kan vi stoppe
maskinens bevægelse. Til sidst ved vi hvornår vi er i origo, altså at
maskinen er nulstillet. Vi kan harefter begynde at anvende
maskinen. Det kræver selvfølgelig,at man placerer sensorerne, således
at mekanikken aldrig når at komme til sine yderpunkter, hvor den ikke
kan bevæge sig længere.

\section{SD-kort adapter}
Vi har vha. kravspecefikationen bestemt, at vi skal kunne styre
maskinen, vha. data fra et SD-kort. Et SD-kort kan styres på
forskellige måder.

\begin{itemize}
\item{SPI mode (seperat serial ind eller serial ud)}
\item{Et-bit SD mode (seperat kommando og data kanaler samt et
    proprietært overførsels format)}
\item{Fire-bit SD mode (bruger ekstra forbindelser samt nogle
    ombyttede pins for at supportere fire bits bredde på parallel
    overførsel)}
\end{itemize}

---SPI mode (separate serial in and serial out), one-bit SD mode
(separate command and data channels and a proprietary transfer
format), and four-bit SD mode (uses extra pins plus some reassigned
pins) to support four bit wide parallel transfers.---\fixme{Det
  engelske skal slettes, og det danske tjekkes igennem for
  formulereing}

\fixme{kjærgaard, kan du ikke verificere at ovenstående er korrekt, og
  evt skrive en begrundelse for hvorfor vi vælger SPI mode?}

\fixme{Skal vi refere captain.at som kilde, eller bare i
  litteraturlisten? pinouts.ru}

Som en følge af hvilken måde vi vælger, at kommunikere med SD-kortet,
skal hardwaren understøtte funktionen. Der er dog ikke stor forskel
SPI mode, som vi vil understøtte og er den simpleste, og SD mode, som
blot kræver to forbindelser mere til SD-kortet.

SD-kortet opererer ved en spænding på $3,3V$, derfor skal vi have en
seperat spændingsforsyning på $3,3V$ til kortet. A samme grund er vi
nødt til, at sænke spændingen fra vores MCU fra de 5V, til omkring de
3,3V, før vi kan kommunikere med det. Vi kan opnå dette med en
spændingsdeler, ved IO pinsene der går fra MCU'en til SD-kortet. Når
signalet fra SD-kortet går høj, vil det givetvis være på 3,3V. Dette
er ikke de 5V som voes MCU opererer ved, men det er ikke noget
problem, da det stadigvæk svarer til et højt signal for MCU'en.

%%% Local Variables: 
%%% mode: latex
%%% TeX-master: "../master"
%%% End: 