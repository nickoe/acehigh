%
% $Id$
%

\documentclass[11pt,a4paper,oneside,article]{memoir}

\usepackage[utf8]{inputenc}
\usepackage[T1]{fontenc}
\usepackage[adobe-utopia]{mathdesign}
\usepackage[scaled]{luximono}

\usepackage[colorlinks]{hyperref}

%\chapterstyle{section}
\setsecnumdepth{chapter}

\title{AceHigh Control Protocol}

\begin{document}

\maketitle

\tableofcontents*

\chapter{Introduction}

Why this paper? Why is it necessary? How should it be read?

This paper describes the AceHigh Control Protocol used for controlling
an AceHigh plotter.

This protocol is influenced by the RepRap Generation 3 Protol (see
\url{http://docs.google.com/Doc?id=dd5prwmp_14ggw37mfp&hl=en}).

This is a draft and everything in this paper is subject to change.


\chapter{Communication model}

This protocol is intended for communication between two nodes: an
intelligent node instantiating the connection, making various
decisions based on the device capabilities e.g., and an unintelligent
node answering requests and notifying the host when certain event
occur.

Put an illustration of the communication model here!

Each node can have one session open at a time. This means, that both
nodes can send request and recieve responses, but a node must get the
response before sending a new request.


\chapter{Packet structure}

The packet structure is shown in Figure~\ref{fig:packet-structure}.

\begin{figure}[htbp]
  \centering
  \vspace{2cm}
  \caption{Packet structure}
  \label{fig:packet-structure}
\end{figure}


\section{Synchronization byte}

Each packet is preceded by a synchronization byte. This byte has the
value of $\textup{D}5_{16}$ (or $213_{10}$ or $11010101_2$).


\section{Header byte}

One byte describing the purpose of the packet. The first byte
indicated whether the packet is a request (0) or a response (1).


\section{Data bytes}

The number of data bytes depends on the header byte. Each header byte
has a fixed number of data bytes.


\section{Checksum}

The checksum is the CRC-8-Dallas/Maxim of the header and data. The
synchronization bit is not included in the checksum byte.


\chapter{Request codes}

This section lists all request codes. All numbers are base 10.


\section{0 -- Get firmware version}

This requests the firmware version of the device.


\section{127 -- Extended requests}

This request is reserved for future versions of this protocol.


\chapter{Response codes}


\chapter{Revision history}

This section will be written when the first stable version of this
protocol is designed.


\end{document}
