% Retningslinier og vejledning til projekt
%

\documentclass[simple,final]{../mypaper}

\usepackage{svn}
\SVN $Id$

\usepackage[scaled]{luximono}

\usepackage{listingsutf8}

\lstset{
  language=[LaTeX]TeX,
  extendedchars=true,
  inputencoding=utf8/latin1,
}

\title{Retningslinier og vejledninger}
\date{SVN-id: \SVNId}

\hypersetup{
  pdftitle={\thetitle},
  %pdfsubject={},
  pdfauthor={Christian Klim Hansen, Kristian Kjærgaard og Nick Østergaard},
  pdfcreator=LaTeX
}

\begin{document}

%\maketitle
{
\Huge \flushright \bfseries \thetitle
}
{
\flushright SVN-id: \SVNId
}


\tableofcontents*


\section{Indledning}

Her står der hvad vi bruger guiden til.



\chapter{Gruppekontrakt}

Følgende kontrakt indeholder retningslinjer, som skal overholdes for
at give et optimalt arbejdsproces.


\section{Pauser}

Alle gruppemedlemmer har krav på de pauser, som er blevet anvist af
skolen. Tidspunktet hvor disse bliver afholdt kan være
varierende. Pauserne kan også bestå i et par minutters stilhed for at
samle tankerne om projektet, hvilket ligeledes er acceptabelt.


\section{Mødetider}

De normale mødetider er kl. 10:30, hvor ALLE medlemmer fra gruppen
skal befinde sig i klasseværelset (hvis muligt). Evt. sygdom kan godt
undskyldes.


\section{Disciplinære problemer}

Dette punkt hænger meget sammen med kontakt imellem medlemmerne, da vi
ønsker, at folk laver deres ting til tiden og overholder aftaler, som
er lavet imellem de forskellige medlemmer (se specielle aftaler). Hvis
man ikke får lavet det man skal, så giver man en øl i lufthavnen. De
andre medlemmer skal også informeres, hvis man ikke har lavet sine
ting.


\section{Kontakt imellem medlemmerne}

Vi arbejder på at kontaktet imellem medlemmer er god. Vigtige
ændringer skal indberettes til alle. Hvis ikke, se specielle
aftaler. Kontaktinformation fremgår af tabel\vref{tab:kontaktinfo}.

\begin{table}[htbp]
  \centering
  \caption{Kontaktinformation}
  \label{tab:kontaktinfo}
  \begin{tabular}{lll}
    \hline
    Navn & Telefonnummer & E-mail-adresse \\
    \hline
    Christian Klim Hansen & 26 12 19 04  & \url{christian.klim.hansen@gmail.com} \\
    Kristian Kjærgaard & 28 96 36 28  & \url{kkjaergaard@gmail.com} \\
    Nick Østergaard & 20 86 01 20  & \url{oe.nick@gmail.com} \\
    \hline
  \end{tabular}
\end{table}


\section{Materialer}

Følgende er materialer som vi skal have med til hver gang:

\begin{itemize}
\item Kamera (kun 1 stk.)
\item Papir og blyant
\end{itemize}


\section{Specielle aftaler}

Det behøves ikke at informeres om specielle aftaler imellem to
gruppemedlemmer som ikke direkte påvirker andre medlemmer i
gruppe. Det er dog vigtigt, at sygdom og andre forhindringer for
fremmøde indberettes i god tid til alle medlemmer.


\chapter{Vejledninger til \LaTeX}

Her følger en kort vejledning til hvordan de mest almindelige ting
gøres i \LaTeX .

\section{Indsæt figur}

Eksemplerne i dette afsnit viser, hvordan man sætter figurer ind i
\LaTeX . \textbf{Husk at angive billedfil uden filtype (f.eks. .jpeg),
  og husk at lave en label med mening.} Se afsnit om navngivning af
labels. Læg billeder i mappestrukturen som beskrevet i afsnittet
derom.

\lstinputlisting[caption={Eksempel på flydende objekt,
  f.eks. billede. Objektet indsættes nær placeringen i kilden. Filen
  findes i
  \texttt{eksempler/figure.tex}.},label={lst:figure}]{./eksempler/figure.tex}

\lstinputlisting[caption={Eksempel på objekt i margin,
  f.eks. billede. Eksemplet findes i
  \texttt{eksempler/figure-margin.tex}.},label={lst:figure-margin}]{./eksempler/figure-margin.tex}


\section{Indsæt kodeeksempel}

Eksemplerne i dette afsnit viser, hvordan man indsætter kodeeksempler
i \LaTeX . Læg alle kodeeksempler i mappestrukturen som beskrevet i
afsnittet derom.

\lstinputlisting[caption={Eksempel på løbende kodeeksempel. Eksemplet
  findes i
  \texttt{eksempler/kode.tex}.},label={lst:kode}]{./eksempler/kode.tex}


\section{Lav henvisning}

Eksemplerne viser, hvordan man laver henvisning til labels.

\lstinputlisting[caption={Henvisning til
  figur-/kodeeksempel-/tabelnummer med sidetal. Eksemplet findes i
  \texttt{eksempler/vref.tex}},label={lst:vref}]{./eksempler/vref.tex}

Makroen \texttt{\textbackslash vref} kan give problemer. I dette tilfælde
bør man gøre som beskrevet i kodeeksempel\vref{lst:ref}.

\lstinputlisting[caption={Henvisning til
  figur-/kodeeksempel-/tabelnummer uden sidetal. Bruges når vref giver
  problemer. Eksemplet findes i
  \texttt{eksempler/ref.tex}},label={lst:ref}]{./eksempler/ref.tex}


\section{Navngivning af labels}

Navngiv labels efter følgende regler:

\begin{itemize}
\item Brug små bogstaver
\item Brug præfikset \texttt{fig:} ved figurer, \texttt{tab:} ved
  tabeller og \texttt{lst:} til kodeeksempler.
\item Brug bindesteg (-) til at adskille navne i en label,
  f.eks. \texttt{\textbackslash label\{fig:data-temp\}}
\item Tilstræb et kort navn der giver mening.
\end{itemize}


\section{Brug af \texttt{\textbackslash fixme}}

En \texttt{\textbackslash fixme} er en markering af en fejl, som skal
rettes inden rapporten afleveres. Eksempel\vref{lst:fixme} viser,
hvordan man indsætter en \texttt{\textbackslash fixme}.

\lstinputlisting[caption={Eksempel på en \texttt{\textbackslash
    fixme}. Eksemplet findes i
  \texttt{eksempler/fixme.tex}},label={lst:fixme}]{./eksempler/fixme.tex}

Lav \texttt{\textbackslash fixme}s efter følgende regler:

\begin{itemize}
\item Skriv navnet til den, noten er henvendt til
  (f.eks. \texttt{\textbackslash fixme\{Klim: er afsnittet
    konsistent?\}})
\end{itemize}


\chapter{Vejledning til Subversion (SVN) og filhåndtering}

\section{Arbejdsgang for arbejde med materiale på SVN-serveren}

Arbejdsgangen for arbejde med materiale på SVN-serveren er:

\begin{enumerate}
\item Opdater lokalt arkiv (\texttt{svn update}).
\item Udfør arbejde.
\item Opdater lokalt arkiv igen (\texttt{svn update}) og løs
  konflikter.
\item Upload ændringer til offentligt arkiv (\texttt{svn commit}).
\end{enumerate}

Vær opmærksom på følgende:

\begin{itemize}
\item Skriv en meddelelse, når du uploader.
\item Tilføj alle filer, der er nødvendige for at andre kan arbejde
  videre. Tilføj gerne filerne når de oprettes for at huske det.
\item Tilføj kun ét format af hver fil (f.eks. kun billede.svg og ikke
  billede.pdf).
\end{itemize}


\section{Filplacering og -navngivning}

Navngiv filer efter følgende regler:

\begin{itemize}
\item Giv filen et navn med mening
  (f.eks. \texttt{software-sd.tex} for software til
  SD-kort-læseren)
\item Brug små bogstaver
\item Brug bindestreg (-) til at adskille navne. Mellemrum er ikke tilladt
\item æ, ø og å er tilladt
\end{itemize}

Placer filerne i mapper efter følgende mønster:

\begin{itemize}
\item \texttt{software/} indeholder software til MCU'en
\item \texttt{hardware/} indeholder hardwaredokumentation
\item \texttt{logo/} indeholder projektets logo, som skal findes på
  forsiden af rapporten
\end{itemize}


\end{document}
