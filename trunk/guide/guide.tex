% Retningslinier og vejledning til projekt
%

\documentclass[simple,final]{../mypaper}

\usepackage{svn}
\SVN $Id$

\usepackage[scaled]{luximono}

\usepackage{listingsutf8}

\lstset{
  language=[LaTeX]TeX,
  extendedchars=true,
  inputencoding=utf8/latin1,
  basicstyle=\footnotesize \ttfamily
}

\title{Retningslinier og vejledninger}
\date{SVN-id: \SVNId}

\begin{document}

%\maketitle
{
\Huge \flushright \bfseries \thetitle
}
{
\flushright SVN-id: \SVNId
}


\tableofcontents*


\section{Indledning}

Her står der hvad vi bruger guiden til.



\chapter{Vejledninger til \LaTeX}

Her følger en kort vejledning til hvordan de mest almindelige ting
gøres i \LaTeX .

\section{Indsæt figur}

Eksemplerne i dette afsnit viser, hvordan man sætter figurer ind i
\LaTeX . \textbf{Husk at angive billedfil uden filtype (f.eks. .jpeg),
  og husk at lave en label med mening.} Se afsnit om navngivning af
labels. Læg billeder i mappestrukturen som beskrevet i afsnittet
derom.

\lstinputlisting[caption={Eksempel på flydende objekt,
  f.eks. billede. Objektet indsættes nær placeringen i kilden. Filen
  findes i
  \texttt{eksempler/figure.tex}.},label={lst:figure}]{./eksempler/figure.tex}

\lstinputlisting[caption={Eksempel på objekt i margin,
  f.eks. billede. Eksemplet findes i
  \texttt{eksempler/figure-margin.tex}.},label={lst:figure-margin}]{./eksempler/figure-margin.tex}


\section{Indsæt kodeeksempel}

Eksemplerne i dette afsnit viser, hvordan man indsætter kodeeksempler
i \LaTeX . Læg alle kodeeksempler i mappestrukturen som beskrevet i
afsnittet derom.

\lstinputlisting[caption={Eksempel på løbende kodeeksempel. Eksemplet
  findes i
  \texttt{eksempler/kode.tex}.},label={lst:kode}]{./eksempler/kode.tex}


\section{Lav henvisning}

Eksemplerne viser, hvordan man laver henvisning til labels.

\lstinputlisting[caption={Henvisning til
  figur-/kodeeksempel-/tabelnummer med sidetal. Eksemplet findes i
  \texttt{eksempler/vref.tex}},label={lst:vref}]{./eksempler/vref.tex}

Makroen \texttt{\textbackslash vref} kan give problemer. I dette tilfælde
bør man gøre som beskrevet i kodeeksempel\vref{lst:ref}.

\lstinputlisting[caption={Henvisning til
  figur-/kodeeksempel-/tabelnummer uden sidetal. Bruges når vref giver
  problemer. Eksemplet findes i
  \texttt{eksempler/ref.tex}},label={lst:ref}]{./eksempler/ref.tex}


\section{Navngivning af labels}

Navngiv labels efter følgende regler:

\begin{itemize}
\item Brug små bogstaver
\item Brug præfikset \texttt{fig:} ved figurer, \texttt{tab:} ved
  tabeller og \texttt{lst:} til kodeeksempler.
\item Brug bindesteg (-) til at adskille navne i en label,
  f.eks. \texttt{\textbackslash label\{fig:data-temp\}}
\item Tilstræb et kort navn.
\end{itemize}


\section{Brug af \texttt{\textbackslash fixme}}

Hvordan bruger vi vores fixmes?


\end{document}
